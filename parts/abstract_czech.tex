\begin{czech}
\begin{abstract}

Párování prohledává možné páry nabídky a poptávky, které řadí dle míry, s jakou nabídka vyhovuje poptávce.
Tato práce demonstruje, jak lze dva obecné postupy, jmenovitě případové usuzování a statistické relační učení, použít pro párování veřejných zakázek a uchazečů o zakázky.
V obou případech párování využívá jak logické, tak statistické usuzování operující ve vzájemně porovnatelných, polo-strukturovaných a sémanticky popsaných datech.
Na základech případového usuzování jsme navrhli novou metodu párování implementovanou pomocí dotazovacího jazyka SPARQL pro data ve formátu RDF.
Metoda využívá podobnostní vyhledávání učící se z dříve udělených zakázek, které interpretuje jako zkušenosti vyřešených problémů.
Pro párování vycházející ze statistického relačního učení jsme převzali RESCAL, což je algoritmus pro faktorizaci multi-relačních tenzorů využívající kolektivní učení pro predikci vazeb.
Náš přínos v obou přístupech zahrnuje zejména výběr a tvorbu příznaků a také ladění parametrů párování.

Metody párování jsme aplikovali na soubor propojených otevřených dat veřejné správy, jehož ústředním prvkem je Věstník veřejných zakázek.
Doménu veřejných zakázek jsme zvolili, protože poskytuje explicitně popsané poptávky, které jsou díky zákonem vyžadovanému proaktivnímu zveřejňování oznámení o veřejných zakázkách dostupné v podobě otevřených a strukturovaných dat.
Náš výzkum je motivován rozsáhlým pasivním plýtváním ve veřejných zakázkách, které má párování šanci zmírnit návrhy efektivnější alokace veřejných prostředků.
Věstník veřejných zakázek jsme pro účely párování integrovali s dalšími daty veřejné správy, jako jsou číselníky nebo rejstříky právních osob.
Příprava dat si vyžádala rozsáhlé úsilí při budování komplexních ETL procesů, jednak z důvodu mnoha problémů kvality dat o veřejných zakázkách, ale také kvůli nesourodosti kombinovaných datových sad.
Jako rámec datové integrace jsme využili propojená otevřená data, která staví na technologických standardech sémantického webu.
Řešení klíčových problémů dat zahrnovalo především návrh a implementaci technik pro propojování a fúzi dat.
V průběhu přípravy dat jsme otestovali a integrovali dostupný software založený na technologiích sémantického webu, ale také vyvinuli přepoužitelné nástroje pro předzpracování dat ve formátu RDF.

Evaluaci metod párování jsme provedli na úloze predikce vítězných uchazečů o zakázky v retrospektivních datech o zakázkách udělených během doby 10 let.
Evaluací metrik přesnosti a diverzity jsme vyhodnotili přínos dílčích faktorů ovlivňujících párování, jako je například expanze dotazů nebo objem dat pro strojové učení.
Kvalita a rozsah vstupních dat se projevily jako zásadní faktory rozhodující o úspěšnosti párování.
Párování využívající SPARQL ve všech ohledech jednoznačně překonalo přístup založený na algoritmu RESCAL, a to zejména s ohledem na diverzitu výsledků a náročnost výpočtu.
Na rozdíl od většiny využitých příznaků, které se projevily jako šum, se příznaky z řízených slovníků popisujících zakázky nebo uchazeče ukázaly pro párování jako podstatně informativnější.
Na hodnotu propojených dat poukázaly nejlepší výsledky u obou přístupů, které byly dosaženy párováním kombinujícím příznaky z více datových zdrojů.

\emph{Klíčová slova:} párování, propojená data, otevřená data, veřejné zakázky

\end{abstract}
\end{czech}
